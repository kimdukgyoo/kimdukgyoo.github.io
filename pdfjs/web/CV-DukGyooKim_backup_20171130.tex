% LaTeX file for resume 
% This file uses the resume document class (res.cls)
\documentclass[margin]{res} 
\usepackage{hyperref}
\usepackage{multicol}
\hypersetup{colorlinks=true, linkcolor=blue,  anchorcolor=blue, citecolor=blue, filecolor=blue, menucolor=blue, urlcolor=blue} 
\usepackage{inputenc}
\usepackage{fontenc}
% the margin option causes section titles to appear to the left of body text 
\textwidth=5in % increase textwidth to get smaller right margin
%\usepackage{helvetica} % uses helvetica postscript font (download helvetica.sty)
%\usepackage{newcent}   % uses new century schoolbook postscript font 

\begin{document} 
\name{\Large{Duk Gyoo Kim}\\[12pt]} % the \\[12pt] adds a blank line after name
%\address{467 Uris Hall, Cornell University\\Ithaca, NY 14853 USA}
%\address{Phone: (607) 793-6539\\ E-mail: \href{mailto:dk626@cornell.edu}{dk626@cornell.edu} }

\begin{resume} 
\section{Contact Information} Office: L7, 3-5, Room 225, 68161 Mannheim, Germany\\
Phone: +49-621-181-1797 (Office), +49-157-7482-7036 (Mobile)\\
Email: \href{mailto:d.kim@uni-mannheim.de}{d.kim@uni-mannheim.de}\\
Webpage: \url{kimdukgyoo.github.io}

\section{Employment}
\textbf{Assistant Professor}, Department of Economics, University of Mannheim, Sept. 2017--\\
\textbf{Postdoctoral Scholar}, California Institute of Technology, Oct. 2015--Sept. 2017%\\
%(* Dan Searle Fellowship in Economics/ Mentor: Thomas R. Palfrey)

\section{Fields}
\textbf{Primary}: Public Economics, Political Economy, Experimental Economics\\
\textbf{Secondary}: Microeconomic Theory, Behavioral Economics

\section{Education} 
\textbf{Ph.D. Economics,} Cornell University, Aug. 2015\\
%$\phantom{aa}$- Dissertation title: Essays on Applied Economics with Experimental Evidence\\
%$\phantom{aa}$- Advisors: Robert H. Frank, Stephen Coate, William D. Schulze\\
%\textbf{M.A. in Economics,} Cornell University, Sep. 2013\\
\textbf{M.A. in Economics,} Yonsei University, Seoul Korea,  Aug. 2010\\
\textbf{B.A. in Business Administration,} Yonsei University, Seoul Korea, Feb. 2008
%
%\section{Job Market Papers}
%\textbf{``One Bite at the Apple": Legislative Bargaining without Replacement} (Job Market Paper \#1)\\
%: I modify the many-player ``divide-the-dollar'' game in which previous proposers, players who were randomly selected in the previous rounds but failed to provide an accepted proposal, cannot propose again. This bargaining model without replacement captures the captures the legislative process where each legislator has only one chance of opportunity, and yields a unique subgame perfect equilibrium which has two distinctive features: Under majority or unanimity, the first proposer keeps a constant share for herself \emph{regardless} of the size of the legislature. Under unanimity, the first proposer keeps a \emph{smaller} share than the nonproposers when the discount factor is sufficiently large. Because of these features, the behavioral factors that could be driving the bargaining outcomes in the laboratory, such as retaliation and a concern about fairness, can be identified. I find that proposers do not fully extract their rent, but that a concern about fairness is not a driving factor at all. Out-of-equilibrium observations suggest that retaliation and the fear thereof can be a driving factor.\\\\
%\textbf{The Cycle-Stationary Subgame Perfect Equilibrium in Legislative Bargaining without Replacement} (Job Market Paper \#2)\\
%: This paper studies infinite-horizon sequential bargaining among n$\ge$3 players in which the proposer---a player who proposes a distribution of an economic surplus---is randomly selected from the pool of potential proposers. If the proposal is rejected, the current and previous proposers are excluded from the pool of potential proposers and the game moved on to the next round, until every player has had the same number of opportunities to be the proposer. To analyze the infinite-horizon model with a particular time dependency within each sequence of $n$ rounds (a cycle,) I address the equilibrium characterization with an extended notion of stationarity, which I call \emph{cycle-stationary subgame perfect} (CSSP) equilibrium. The CSSP equilibrium is unique, and it is analogous to the subgame perfect equilibrium of some specific forms of finite-horizon bargaining. The proposer's share in the CSSP equilibrium is strictly smaller than that predicted by the stationary equilibrium of the Baron--Ferejohn legislative bargaining model under any voting rule except unanimity. 

\section{Working Papers}
\begin{enumerate}
\item ``One Bite at the Apple": Legislative Bargaining without Replacement
\item The Cycle-Stationary Subgame Perfect Equilibrium in Legislative Bargaining without Replacement
\item Public Announcement under Rational Ignorance (with Yeochang Yoon)
%\item Legislative Bargaining over Particularistic and Collective Goods Revisited
%\item Economic Conservatism of the Poor: Low Demand for Redistribution
%\item The Coin Strategy and Charitable Giving
\end{enumerate}

\section{Work in Progress}
\begin{enumerate}
\item Multilateral Bargaining with Proposer Selection Contests (with Sang-Hyun Kim)
\item Mixing Propensity and Strategic Decision Making (with Hee Chun Kim)
%\item The Salience of Positive Externality of Vaccination
\end{enumerate}

\section{Publications}
\begin{enumerate}
\item Population Uncertainty in Voluntary Contributions of Public Goods, \emph{Journal of Economic Behavior \& Organization}, 2018, Volume 145, 218--231
\item The Second-Tier Trap: Theory and Experimental Evidence, forthcoming at the \emph{International Journal of Economic Theory}
\item Response time in choosing the most or least preferred option, \textit{Economics Bulletin}, 2016, Vol. 36 No. 1 pp. 595--600
\item Why Are the Poor Conservative? (with Paul Moon Sub Choi) \textit{The Korean Journal of Economics}, 2015, Vol 22(1), pp. 15--24
\end{enumerate}

%
%\section{Referred Publications (Non-SSCI)}
%\begin{enumerate}
%\item Response time in choosing the most or least preferred option, \textit{Economics Bulletin}, 2016, Vol. 36 No. 1 p.A59
%\item Why Are the Poor Conservative? (with Paul Moon Sub Choi) \textit{The Korean Journal of Economics}, 2015, Vol 22(1), pp. 15--24
%\end{enumerate}
%\section{Research Experience}
%\begin{itemize}
%\item RA to Prof. J\"{o}rg Stoye, Department of Economics, Cornell University, September 2013 to August 2015
%\item RA to Dr. Dean Lillard and Dr. Rebekka Christopoulou, Department of Policy Analysis and Management, Cornell University, January to June 2012
%\item RA to Prof. Byung Sam Yoo, service for the National Assembly Budget Settlement Committee, Korea, October 2008 to November 2008
%\end{itemize}

%\section{Teaching Experience}
%\textit{At Cornell University:}
%\begin{itemize}
%\item Intermediate Microeconomics, TA for Prof. Maxim Troshkin, Spring 2014
%\item Econometrics II (PhD core), TA for Prof. J\"{o}rg Stoye, Spring 2012, Spring 2013
%\item Intro to Probability and Statistics, TA for Prof. Yongmiao Hong, Fall 2012
%\item Applied Econometrics, TA for Prof. George Jakubson, Summer 2012
%\item Introduction to Microeconomics, COTA (a leading TA) for Prof. Jennifer Wissink, Fall 2011
%\item Introduction to Macroeconomics, COTA for Prof. Steven Kyle, Fall 2013
%\end{itemize}
%\textit{At Yonsei University:}
%\begin{itemize}
%\item Econometrics I, Financial Econometrics, TA for Prof. Byung Sam Yoo, March 2008 to August 2010
%\item Lecturer, A short course on Statistics and Econometrics, Korea Investment and Securities Co., Ltd., Private Client  Strategy Division, March 2010
%\item Lecturer, A short course on Time-Series Econometrics, Sungshin Women's University, Business Administration Research Institute, February 2009
%\item Econometrics Lab Instructor, Bank of Korea, HR Development Institute, October 2008, April 2009, and October 2009
%\end{itemize}


\section{Fellowships, Grants, Honors, and Awards}
\begin{itemize}
\item University of Mannheim, SFB 884 ``The political economy of reforms" (financed by German Research Foundation, DFG), Project A7 (2018--2022, with Hans Peter Gr{\"u}ner)
\item Dan Searle Postdoctoral Fellowship in Economics, Oct. 2015--Sept. 2017.
\item East Asia Program Research Travel Grant, May 2015.
\item Einaudi Center International Research Travel Award, Mar. 2015.
\item Cornell Population Center Rapid Grant, Mar. 2015.
\item Cornell Graduate School Conference Travel Grant, 2014-2015.
\item Charles Koch Foundation Dissertation Research Grant, 2014-2015
\item The Howard and Abby Milstein Graduate Teaching Award, 2014.
\item Cornell Graduate School Research Travel Grant, Nov. 2014.
\item Hayek Fund for Scholars, The Institute of Humane Studies, 2014, 2015, 2016, 2017.
\item The Institute of Humane Studies Conference and Research Grant, Sept. 2014.
\item Cornell Population Center Rapid Grant, Aug. 2014.
%\item Cornell Population Center Rapid Grant for ``Experiment in Group and Individual Effort Choices on Competitions,'' August 2014.
\item Cornell Graduate School Conference Travel Grant, 2013-2014.
\item Science of Philanthropy Initiative PhD grant, Mar. 2014
%\item Science of Philanthropy Initiative PhD grant for ``The Coin Strategy and Charitable Giving," March 2014
\item Fulbright Graduate Study Award, Aug. 2010--Jul. 2012
%\item Brain Korea 21 Research Assistantship, Ministry of Education, South Korea, August 2008 to February 2010
%\item Citi-KIF (Korean Institute of Finance) financial research paper competition grant (\$3,500), June 2009
%\item Citi-KAIST financial research paper competition grant (\$3,000), March 2008
%\item \textit{Maeil} Business and Economy Newspaper 22nd Economic Thesis Award, October 2007
%\item Provost's Honor, UC San Diego (Exchange Program), Fall 2006 quarter 
%\item Dean's Honor of Distinguished Students, Yonsei University, Spring 2005, and Fall 2005 
\end{itemize}
\section{Conference/ Seminar Presentations}
\textbf{2017}: Max Planck Institute for Tax Law and Public Finance, University of Mannheim, Korea Informational Society Development Institute, Korea Institute of Public Finance, Korea Institute of International Economic Policy, Korea Energy Economics Institute, University of Massachusetts Amherst, Western Political Science Association Conference, Kyung Hee University, SKKU Junior Faculty Research Conference, North American Meeting of the Econometric Society, Western Economics Association International Annual Meetings, Yonsei Young Economists Workshop, KAEA-KIPF Conference, Yonsei University, Midwest Economic Theory Conference, HeiKaMaxY Workshop\\
%\textbf{2017 (*=scheduled)}: Max Planck Institute for Tax Law and Public Finance, University of Mannheim, Korea Informational Society Development Institute, Korea Institute of Public Finance, Korea Institute of International Economic Policy, Korea Energy Economics Institute, University of Massachusetts Amherst, Western Political Science Association Conference, Kyung Hee University, SKKU Junior Faculty Research Conference, North American Meeting of the Econometric Society, Western Economics Association International Annual Meetings, Yonsei Young Economists Workshop, KAEA-KIPF Conference, Yonsei University, Midwest Theory Conference*\\
\textbf{2016}: Chapman University, Southwest Experimental and Behavioral Economics Workshop Annual Conference, Canadian Economics Association Annual Conference, Caltech, Western Economics Association International Annual Meetings, Sogang University, Seoul National University, KDI School of Public Policy and Management, KAEA-KEA International Conference, Cornell University, Canadian Public Economics Group Conference, North-American ESA Conference, Sonoma State University, Midwest Economic Theory Conference\\
\textbf{2015}: The 52nd Annual Meetings of the Public Choice Society, The 85th annual conference of the Southern Economic Association\\
\textbf{2014}: Canadian Economics Association Annual Conference, The Institute of Humane Studies Summer Research Colloquium, KAEA-KEA International Conference, The 9th Economics Graduate Student Conference at Washington University in St. Louis, Canadian Public Economics Group Conference, The 2nd Science of Philanthropy Initiative Conference

%\section{Workshop Participation}
%2014: The second Summer School of the Econometric Society in Seoul (Hanyang University, South Korea)

\section{Referee Service} Journal of Public Economics

\section{Citizenship} South Korea (EU Blue Card)% (J-1 Visa)%South Korea (F-1 Visa since 2012, J-1 from 2010 to 2012)

\section{Affiliations} American Economic Association, Econometric Society, Korean-American Economic Association

\section{Languages} Korean(native), English(fluent)

\section{References} \textbf{Thomas R. Palfrey}\\
Flintridge Foundation Professor of Economics and Political Science\\
1200 E California Blvd. MC 228-77, California Institute of Technology\\
Pasadena, CA 91125\\
Phone: (626) 395-4088\\
Email: \href{mailto:trp@hss.caltech.edu}{trp@hss.caltech.edu}\\\\
\textbf{Stephen Coate}\\
Kiplinger Professor of Public Policy\\
476 Uris Hall, Cornell University\\
Ithaca, NY 14853\\
Phone: (607) 255-1912 \\
Email: \href{mailto:sc163@cornell.edu}{sc163@cornell.edu}\\\\
\textbf{Robert H. Frank}\\
Henrietta Johnson Louis Professor of Management and Professor of Economics\\
327 Sage Hall, Johnson Graduate School of Management, Cornell University\\
Ithaca, NY 14853\\
Phone: (607) 255-8501\\
Email: \href{mailto:rhf3@cornell.edu}{rhf3@cornell.edu}
\begin{flushright}
Last Updated: \today
\end{flushright}
\end{resume} 
\end{document} 



